\documentclass{beamer}

% Tema da Beamer (opções: Boadilla, CambridgeUS, Darmstadt, Goettingen, Madrid, Montpellier, etc.)
\usetheme{Madrid} 
\usecolortheme{beaver} % Esquema de cores

% Pacotes úteis
\usepackage[utf8]{inputenc}
\usepackage[portuguese]{babel} % Suporte para português
\usepackage{graphicx} % Para incluir imagens
\usepackage{amsmath} % Para equações matemáticas
\usepackage{amssymb} % Para símbolos matemáticos
\usepackage{ragged2e} % Para justificar texto
\usepackage{hyperref} % Para links

% Configurações globais
\justifying % Justifica o texto em todos os slides

% Título da apresentação
\title[RCF]{A Radiação Cósmica de Fundo: Eco do Big Bang e Janela para o Universo Primitivo}
\author[Matheus Bregonci Pires]{Matheus Bregonci Pires}
\institute[Física - UFES]{Física - UFES}
\date{02 de Julho de 2025} % Atualize a data se necessário

\begin{document}

% --- Slide de Título ---
\begin{frame}
    \titlepage
\end{frame}

% --- Sumário ---
\begin{frame}{Sumário}
    \tableofcontents
\end{frame}

% --- Seção 1: Introdução ---
\section{Introdução: O que é a RCF?}

\begin{frame}{O que é a Radiação Cósmica de Fundo (RCF)?}
    \begin{itemize}
        \item Você já pensou em como sabemos sobre o início do universo? Qual a evidência mais forte para a teoria do Big Bang?
        \item A RCF é a \textbf{radiação eletromagnética remanescente} do universo primordial.
        \item É o \textbf{"fóssil" mais antigo} do Big Bang, uma "foto" do universo bebê.
        \item \textbf{Por que é importante?}
        \begin{itemize}
            \item É uma das \textbf{evidências mais fortes} para a teoria do Big Bang.
            \item Fornece \textbf{informações cruciais} sobre a idade, composição e evolução do universo.
        \end{itemize}
        \item \textbf{Objetivo:} Compreender a origem, propriedades e o significado cosmológico da RCF.
    \end{itemize}
\end{frame}

% --- Seção 2: Contexto Cosmológico: O Big Bang ---
\section{Contexto Cosmológico: O Big Bang}

\begin{frame}{O Big Bang e o Universo Primitivo}
    \begin{itemize}
        \item \textbf{O Big Bang não é uma explosão no espaço,} mas a \textbf{expansão do próprio espaço}.
        \item \textbf{Evidências chave:}
        \begin{itemize}
            \item Expansão do universo (Lei de Hubble).
            \item Abundância de elementos leves (Hidrogênio, Hélio, Lítio).
            \item A própria Radiação Cósmica de Fundo (RCF).
        \end{itemize}
        \item \textbf{Universo Primitivo:}
        \begin{itemize}
            \item Extremamente \textbf{quente e denso}.
            \item Estado de \textbf{plasma}: elétrons livres e núcleos atômicos.
            \item \textbf{Opacidade:} A luz estava "presa", interagindo constantemente com as partículas carregadas. O universo era \textbf{opaco}.
        \end{itemize}
    \end{itemize}
\end{frame}

% --- Seção 3: O Surgimento da RCF ---
\section{O Surgimento da RCF: Recombinação e Desacoplamento}

\begin{frame}{A Era da Recombinação (Desacoplamento de Fótons)}
    \begin{itemize}
        \item \textbf{Quando?} Aproximadamente \textbf{380.000 anos} após o Big Bang.
        \item \textbf{Temperatura:} Caiu para cerca de \textbf{3000 K}.
        \item \textbf{O que aconteceu?}
        \begin{itemize}
            \item Formação de \textbf{átomos neutros}: elétrons se combinaram com prótons e núcleos de hélio.
            \item $\text{e}^- + \text{p}^+ \rightarrow \text{H}$ (átomo de hidrogênio neutro)
        \end{itemize}
        \item \textbf{Impacto na Luz:}
        \begin{itemize}
            \item Os fótons não interagiam mais com tantas partículas carregadas.
            \item O universo se tornou \textbf{transparente} à luz.
            \item Essa \textbf{luz "libertada"} é a RCF que observamos hoje.
        \end{itemize}
    \end{itemize}
\end{frame}

\begin{frame}{Esquema da Recombinação}
    \begin{center}
        \includegraphics[width=0.5\textwidth]{exemplo_recombinacao.png} 
        
        \captionof{Antes da Recombinação (opaco) e Depois (transparente).}
    \end{center}
    
    \begin{itemize}
        \item No início, o universo era um plasma opaco de partículas carregadas e fótons.
        \item Com o resfriamento, formaram-se átomos neutros, permitindo que os fótons viajassem livremente.
    \end{itemize}
    
\end{frame}


% --- Seção 4: A Descoberta da RCF ---
\section{A Descoberta da RCF}

\begin{frame}{Previsão Teórica}
    \begin{itemize}
        \item \textbf{Décadas de 1940-1960:}
        \item \textbf{George Gamow}, Ralph Alpher e Robert Herman:
        \begin{itemize}
            \item Previram a existência de um fundo de radiação remanescente do Big Bang.
            \item Estimaram uma temperatura de poucos kelvins para essa radiação.
            \item Essa previsão, embora com algumas imprecisões iniciais, foi fundamental.
        \end{itemize}
    \end{itemize}
\end{frame}

\begin{frame}{A Descoberta Acidental (Penzias e Wilson)}
    \begin{columns}
        \column{0.5\textwidth}
        \begin{itemize}
            \item \textbf{1964/1965:} Arno Penzias e Robert Wilson (Bell Labs).
            \item Testavam uma antena de micro-ondas super sensível para comunicação por satélite.
            \item Detectaram um \textbf{"ruído" persistente e isotrópico} vindo de todas as direções do céu.
            \item Inicialmente, pensaram em falhas no equipamento ou sujeira (inclusive "dejetos de pombos"!).
        \end{itemize}
        \column{0.5\textwidth}
        \begin{center}
            \includegraphics[width=0.9\textwidth]{penzias_wilson.jpg} % Substitua pela imagem dos cientistas e antena

            \tiny
            \captionof{A Antena Holmdel Horn se eleva acima de Arno Penzias (à direita) e Robert Wilson, nesta fotografia de 1978, ano em que os dois rádioastrônomos ganharam o Prêmio Nobel de Física. Fonte: National Geographic.}
        \end{center}
    \end{columns}
\end{frame}

\begin{frame}{A Confirmação e o Prêmio Nobel}
    \begin{itemize}
        \item \textbf{Conexão com Princeton:} Penzias e Wilson entraram em contato com o grupo de Robert Dicke e Jim Peebles, que estava procurando por essa radiação.
        \item \textbf{Significado:} A observação confirmou de forma espetacular as previsões da teoria do Big Bang.
        \item \textbf{Prêmio Nobel de Física (1978):} Concedido a Arno Penzias e Robert Wilson pela sua descoberta da RCF.
    \end{itemize}
\end{frame}

% --- Seção 5: Propriedades da RCF ---
\section{Propriedades da RCF}

\begin{frame}{Natureza da RCF: O Espectro de Corpo Negro}
    \begin{itemize}
        \item A RCF é uma \textbf{radiação de micro-ondas}.
        \item \textbf{Espectro de Corpo Negro Quase Perfeito:}
        \begin{itemize}
            \item É o espectro de corpo negro mais perfeito já medido na natureza.
            \item Isso é uma forte evidência de um universo quente e denso no passado.
        \end{itemize}
        \item \textbf{Temperatura Atual:} Aproximadamente \textbf{2,72K}.
        \item \textbf{Por que tão fria?}
        \begin{itemize}
            \item \textbf{Resfriamento devido à expansão do universo (Redshift):}
            \item $\lambda \propto a(t)$ (comprimento de onda se estica com a expansão do universo).
            \item A energia dos fótons diminui, e sua temperatura aparente também.
        \end{itemize}
    \end{itemize}
\end{frame}

\begin{frame}{Gráfico do Espectro de Corpo Negro da RCF}
    \begin{center}
        \includegraphics[width=0.4\textwidth]{espectro_rcf.png}
        
        \tiny
        \captionof{A Radiação de Fundo de Micro-ondas observada pelo observatório espacial COBE. A curva representa um corpo negro de 2,74 K, e os pontos com barras de erro são os dados experimentais. O ajuste é o melhor ajuste experimental de um corpo negro jamais obtido. É sempre bom lembrar que o fato da RCF ter um espectro perfeito de corpo negro não implica que ela seja de origem cósmica como considerado no Modelo Padrão da Cosmologia. A observação original de Arno Penzias e Robert Wilson, os descobridores da RCF, corresponde a um único ponto em λ = 73,5 mm, o qual não estava entre as observações do COBE. Crédito da figura: equipe do COBE.}
    \end{center}
\end{frame}

\begin{frame}{Isotropia e Anisotropias da RCF}
    \begin{columns} % Abre o ambiente de colunas
        \column{0.5\textwidth} % Define a primeira coluna
        \begin{itemize}
            \item \textbf{Isotropia:}
            \begin{itemize}
                \item A RCF é notavelmente \textbf{uniforme em todas as direções} do céu.
                \item Isso sugere que o universo é \textbf{homogêneo e isotrópico} em larga escala (Princípio Cosmológico).
            \end{itemize}
            \item \textbf{Anisotropias (Flutuações):}
            \begin{itemize}
                \item Apesar da uniformidade, existem \textbf{pequenas variações de temperatura} (na ordem de partes por 100.000).
                \item $\Delta T / T \approx 10^{-5}$
            \end{itemize}
        \end{itemize}
        
        \column{0.5\textwidth} % Define a segunda coluna
        \begin{center}
            \includegraphics[width=0.9\textwidth]{mapa_rcf_isotropia.png} % Imagem de um mapa isotrópico da RCF

            \tiny
            \captionof{Mapa das anisotropias da Radiação Cósmica de Fundo (RCF). As pequenas flutuações de temperatura (cores) representam variações de densidade no universo primordial, que deram origem às estruturas cósmicas atuais.}
        \end{center}
    \end{columns} % Fecha o ambiente de colunas
\end{frame}

\begin{frame}{As Anisotropias: Sementes das Estruturas}
    \begin{columns}
        \column{0.5\textwidth}
        \begin{itemize}
            \item \textbf{Importância Crucial:}
            \begin{itemize}
                \item Essas pequenas flutuações são as \textbf{"sementes" gravitacionais} a partir das quais as grandes estruturas do universo (galáxias, aglomerados de galáxias) se formaram ao longo do tempo.
                \item Sem elas, o universo seria um lugar liso, sem estrelas ou galáxias.
            \end{itemize}
            \item A análise dessas flutuações nos dá informações sobre:
            \begin{itemize}
                \item Composição do universo.
                \item Idade do universo.
                \item Geometria do espaço.
            \end{itemize}
        \end{itemize}
        \column{0.5\textwidth}
        \begin{center}
            \includegraphics[width=0.9\textwidth]{mapa_rcf_anisotropias.png} % Imagem de um mapa de anisotropias da RCF (e.g., WMAP ou Planck)

            \tiny
            \captionof{Exemplos de anisotropias. Fonte: D. Saadeh et al ., Phys. Rev. (2016)}
        \end{center}
    \end{columns}
\end{frame}


% --- Seção 6: Missões de Satélites e Seus Resultados ---
\section{Missões de Satélites e Seus Resultados}

\begin{frame}{COBE (Cosmic Background Explorer)}
    \begin{columns}
        \column{0.5\textwidth}
        \begin{itemize}
            \item \textbf{Lançamento:} 1989.
            \item \textbf{Principal Realização:}
            \begin{itemize}
                \item \textbf{Confirmou o espectro de corpo negro quase perfeito} da RCF.
                \item \textbf{Detectou as primeiras anisotropias} (flutuações de temperatura) em larga escala.
            \end{itemize}
            \item \textbf{Prêmio Nobel de Física (2006):} John C. Mather e George F. Smoot, pelo seu trabalho no COBE.
        \end{itemize}
        \column{0.5\textwidth}
        \begin{center}
            \includegraphics[width=0.9\textwidth]{cobe_satellite.jpeg} % Imagem do satélite COBE

            \tiny
            \captionof{Representação do satélite COBE da NASA. Fonte: Feita em IA.}
        \end{center}
    \end{columns}
\end{frame}

\begin{frame}{WMAP (Wilkinson Microwave Anisotropy Probe)}
    \begin{columns}
        \column{0.5\textwidth}
        \begin{itemize}
            \item \textbf{Lançamento:} 2001.
            \item \textbf{Avanços:} Medições muito mais precisas e de maior resolução das anisotropias da RCF.
            \item \textbf{Resultados Chave:}
            \begin{itemize}
                \item Determinação precisa da \textbf{idade do universo}: $\approx 13.8$ bilhões de anos.
                \item \textbf{Composição do universo:}
                \begin{itemize}
                    \item Matéria bariônica: $\approx 4.9\%$
                    \item Matéria escura: $\approx 26.8\%$
                    \item Energia escura: $\approx 68.3\%$
                \end{itemize}
                \item Confirmação de que o universo é \textbf{"plano"} (geometria euclidiana).
            \end{itemize}
        \end{itemize}
        \column{0.5\textwidth}
        \begin{center}
            \includegraphics[width=0.9\textwidth]{wmap_satellite.jpg} % Imagem do satélite WMAP
            \tiny
            \captionof{Representação do satélite WMAP da NASA. Fonte: Feita em IA.}
        \end{center}
    \end{columns}
\end{frame}

\begin{frame}{Planck (Agência Espacial Europeia)}
    \begin{columns}
        \column{0.5\textwidth}
        \begin{itemize}
            \item \textbf{Lançamento:} 2009.
            \item \textbf{Mais Preciso Ainda:} Dados de \textbf{maior resolução e sensibilidade} que o WMAP.
            \item \textbf{Refinou os Parâmetros Cosmológicos:}
            \begin{itemize}
                \item Proporcionou os dados mais precisos da RCF até hoje.
                \item Confirmou a idade e composição do universo com ainda mais exatidão.
            \end{itemize}
            \item \textbf{Desafios:} Algumas pequenas anomalias que ainda estão sendo investigadas, inspirando novas pesquisas.
        \end{itemize}
        \column{0.5\textwidth}
        \begin{center}
            \includegraphics[width=0.9\textwidth]{planck_satellite.jpg} % Imagem do satélite Planck
            
            \tiny
            \captionof{Representação do satélite Planck da ESA. Fonte: Feita em IA.}
        \end{center}
    \end{columns}
\end{frame}

% --- Seção 7: Implicações e Futuro da Pesquisa ---
\section{Implicações e Futuro da Pesquisa}

\begin{frame}{Implicações Cosmológicas da RCF}
    \begin{itemize}
        \item \textbf{Evidência Robusta para o Big Bang:} A RCF é a "pedra fundamental" da cosmologia moderna, confirmando a teoria.
        \item \textbf{Janela para o Universo Primitivo:} Permite-nos estudar o universo em seus primeiros momentos (quase 13.8 bilhões de anos atrás).
        \item \textbf{Teste para Modelos Cosmológicos:} A precisão das medições da RCF permite testar e refinar modelos do universo, como a inflação cósmica.
    \end{itemize}
\end{frame}

\begin{frame}{Novas Perguntas e Direções Futuras}
    \begin{itemize}
        \item \textbf{Polarização da RCF:}
        \begin{itemize}
            \item Busca por modos B primordiais (assinatura de ondas gravitacionais primordiais).
            \item Evidência direta da época da inflação cósmica.
        \end{itemize}
        \item \textbf{Anomalias na RCF:} O que elas nos dizem sobre a física fundamental e além do Modelo Padrão?
        \item \textbf{Novas Missões e Observatórios:}
        \begin{itemize}
            \item Telescópios terrestres (e.g., Atacama Cosmology Telescope - ACT, South Pole Telescope - SPT).
            \item Possíveis futuras missões espaciais para medir a RCF com ainda maior precisão.
        \end{itemize}
    \end{itemize}
\end{frame}

% --- Seção 8: Conclusão ---
\section{Conclusão}

\begin{frame}{Conclusão}
    \begin{itemize}
        \item A \textbf{Radiação Cósmica de Fundo (RCF)} é o \textbf{eco distante do Big Bang}, uma luz de quase 13.8 bilhões de anos.
        \item Sua \textbf{descoberta acidental} por Penzias e Wilson revolucionou a cosmologia.
        \item Suas \textbf{propriedades} (espectro de corpo negro, temperatura, isotropia, anisotropias) nos contam sobre a história, idade e composição do nosso universo.
        \item \textbf{Missões espaciais} como COBE, WMAP e Planck refinaram nosso conhecimento sobre o universo primordial e seus parâmetros.
        \item A RCF continua sendo uma \textbf{ferramenta essencial} para testar modelos cosmológicos e desvendar os mistérios da origem e evolução do cosmos.
    \end{itemize}
\end{frame}

\section{Referências}
\begin{frame}[allowframebreaks]{Referências}
    \nocite{*} % Isso garante que todas as referências citadas no documento (mesmo que apenas uma vez) apareçam aqui
    \bibliographystyle{plain} % Ou um estilo de bibliografia mais adequado para cosmologia (ex: unsrt, abbrv, ieeetr)
    \begin{thebibliography}{99} % O número 99 é apenas para reservar espaço para o alinhamento, ajuste se tiver muitas refs.

        \tiny
        \bibitem{madejsky_v2}
        MADEJSKY, Rainer Karl. \textit{Curso Básico de Astrofísica e Cosmologia: Volume 2 - Das galáxias aos quasares}. Vitória, Editora UFES, 2015.

        \bibitem{liddle_cosmology}
        LIDDLE, Andrew. \textit{An Introduction to Modern Cosmology}. 3rd ed. Chichester: John Wiley & Sons, 2015. % Adicionei edição e editora, se tiver, é bom

        \bibitem{saadeh_2016}
        SAADEH, D. et al. "Planck 2015 results. XIV. Dark energy and modified gravity". \textit{Physical Review D}, v. 94, n. 10, 104005, 2016. % Formato mais completo para artigo. Verifique o número do artigo (104005).

        \bibitem{irsa_planck}
        IRSA. \textit{Planck 2015 Data Release}. Disponível em: \url{https://irsa.ipac.caltech.edu/data/Planck/release_2/docs/cmbo.html}. Acesso em: 1 de julho de 2025. % Data de acesso é importante para sites

        \bibitem{lilith_ufmg_planck}
        LILITH.FISICA.UFMG. \textit{Espectro da Radiação Cósmica de Fundo}. Disponível em: \url{https://lilith.fisica.ufmg.br/dsoares/planck/espectro-rfm.html}. Acesso em: 1 de julho de 2025.

        \bibitem{nasa_wmap}
        NASA. \textit{NASA Announces New WMAP Image of Infant Universe}. Disponível em: \url{https://www3.nasa.gov/home/hqnews/2003/feb/HP_news_03064.html}. Acesso em: 2 de julho de 2025.

        \bibitem{national_geographic}
        NATIONAL GEOGRAPHIC. \textit{Como a generosidade de um estranho na Segunda Guerra nos deu a teoria do Big Bang}. Disponível em: \url{https://www.nationalgeographicbrasil.com/historia/2022/06/como-a-generosidade-de-um-estranho-na-segunda-guerra-nos-deu-a-teoria-do-big-bang}. Acesso em: 2 de julho de 2025.
        
    \end{thebibliography}
\end{frame}

\begin{frame}{Obrigado!}
    \begin{center}
        \Huge Abraços e céus limpos!
        \includegraphics[width=0.8\textwidth]{ocultação 2.png} % Imagem final de um universo ou galáxia
    \end{center}
\end{frame}

\end{document}
